\label{index_md_C__git_with_Fork_Leakman_README}%
\Hypertarget{index_md_C__git_with_Fork_Leakman_README}%
Lien sur le WIki du projet \+: \href{https://gitlab-etu.ing.he-arc.ch/isc/2021-22/niveau-2/2282-1-projet-p2-il-sp/g6/-/wikis/home}{\texttt{ https\+://gitlab-\/etu.\+ing.\+he-\/arc.\+ch/isc/2021-\/22/niveau-\/2/2282-\/1-\/projet-\/p2-\/il-\/sp/g6/-\//wikis/home}}

Description \+: Le jeu se présente sous la forme suivante. Un stickman (bonhomme bâton) est lâché dans un niveau en 2D et se met à avancer tout droit et sans s\textquotesingle{}arrêter. Le joueur ne peut pas contrôler le personnage, mais peut (comme dans une partie de lemmings) altérer le niveau de façon à ce que le stickman arrive à la fin du niveau.

Un éditeur de niveau est intégré au jeu afin que les joueurs puissent créer leurs propres niveaux avant d\textquotesingle{}y jouer.

Le jeu est projeté sur un mur et, pour compléter le niveau, le joueur doit jeter des aimants colorés sur le mur. Le jeu interprète, à l\textquotesingle{}aide d\textquotesingle{}une webcam, chaque aimant de couleur comme un bloc auquel est associé un effet. Par exemple, un bloc rouge fait sauter le stickman. 